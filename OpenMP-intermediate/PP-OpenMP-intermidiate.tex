\documentclass{beamer}
\usetheme{Warsaw}

\usepackage{color}
\usepackage{listings}
\usepackage{url}
\usepackage{booktabs}
\usepackage{pgf}
\usepackage{multicol}
\usepackage{algorithm}
\usepackage{algorithmic}


% global setting

\newcommand{\comment}[1]{}

% list environment gloabal setting

% to specify the size of fonts in lst environment

\lstset{showstringspaces=false}
\lstset{language=C}
\lstset{basicstyle=\tt}
\lstset{keywordstyle=\color{blue}}
\lstset{emphstyle=\color{red}}
\lstset{numbers=left}
\lstset{frame=single}
\lstset{rangeprefix=\/*\ }
\lstset{rangesuffix=\ *\/}
\lstset{includerangemarker=false}

%% slides related macros

\newenvironment{mytitle}[2]
{\begin{frame}{#1} \vspace{0.2\textheight}{\huge #2} \end{frame}}

%% Macro for non-numbered frames listing 
%% \nonumberlisting{title}{file}{options}

\newcommand{\nonumberlisting}[3]
{\noindent{#1}
{\lstinputlisting[#3]{#2}}}

%% Program listing macros

%% \programlisting{file}{desp}{font}
%% \programlistingfirst{file}{desp}{font}{start}{end}
%% \programlistingrest{file}{descp}{font}{start}{end}


%% List a program in one page with a font size.

\newcommand{\programlisting}[3]
{\renewcommand{\lstlistingname}{#3 \bf Example}
\lstinputlisting[basicstyle={#3 \tt},caption={#3 \bf (#1) #2}]{#1}}

%% List a program for the first page within a given range.

\newcommand{\programlistingfirst}[5]
{\renewcommand{\lstlistingname}{#3 \bf Example}
\lstinputlisting[basicstyle={#3 \tt},linerange={#4-#5},caption={#3 \bf (#1) #2}]{#1}}

%% List a program for the second page (or later).

\newcommand{\programlistingrest}[5]
{\lstinputlisting[basicstyle={#3 \tt},linerange={#4-#5}]{#1}}

%% List a program in two pages.

\newcommand{\programlistingtwoslides}[6]
{\begin{frame}                  % do not add [fragile]
\programlistingfirst{#1}{#2}{#3}{#4}{#5}
\end{frame}
\begin{frame}
\programlistingrest{#1}{#2}{#3}{#5}{#6}
\end{frame}}

%% List a program in three pages.

\newcommand{\programlistingthreeslides}[7]
{\begin{frame}                  % do not add [fragile]
\programlistingfirst{#1}{#2}{#3}{#4}{#5}
\end{frame}
\begin{frame}
\programlistingrest{#1}{#2}{#3}{#5}{#6}
\end{frame}
\begin{frame}
\programlistingrest{#1}{#2}{#3}{#6}{#7}
\end{frame}
}


%% Program input/output macros

%% \programoutput{file}
%% \programoutputfirst{file}{font}{options}
%% \programoutputrest{file}{font}{startline}{endline}
%% \programoutputtwoslides{file}{font}{start}{endl}{start2}{end2}

%% Program output macor with default setting

\newcommand{\programoutput}[1]
{\programoutputfirst{#1}{}{}}

%% Program output macor for the first page

\newcommand{\programoutputfirst}[3]
{\lstset{basicstyle={#2 \tt}}
\nonumberlisting{#2 \bf 輸出}{#1.out}{#3}}

%% Program output macor for later page

\newcommand{\programoutputrest}[4]
{\lstinputlisting[firstnumber={#3},basicstyle={\tt #2},linerange={#3-#4}]
{#1.out}}

%% Program output macor for two slides

\newcommand{\programoutputtwoslides}[6]
{\begin{frame}                  % do not add [fragile]
\programoutputfirst{#1}{#2}{linerange={#3-#4}}
\end{frame}
\begin{frame}
\programoutputrest{#1}{#2}{#5}{#6}
\end{frame}}


%% \programinput{file}
%% \programinputfirst{file}{font}{option}

%% Program input macor with default setting

\newcommand{\programinput}[1]
{\programinputfirst{#1}{}{}}

%% Program input macor for the first page

\newcommand{\programinputfirst}[3]
{\lstset{basicstyle={#2 \tt}}
\nonumberlisting{#2 \bf 輸入}{#1.in}{#3}}


%% \piotwocol{file}

\newcommand{\piotwocol}[1]{\piotwocoldetail{#1}{}{}}

\newcommand{\piotwocoldetail}[2]{
\begin{columns}[t,onlytextwidth]
\begin{column}{0.45\textwidth}
\programinputfirst{#1}{#2}{}
\end{column}
\begin{column}{0.45\textwidth}
\programoutputfirst{#1}{#2}{}
\end{column}
\end{columns}}

%% \piotworow{file}

\newcommand{\piotworow}[1]
{\programinput{#1}
\programoutput{#1}}

\newcommand{\piotworowdetail}[2]
{\programinputfirst{#1}{#2}{}
\programoutputfirst{#1}{#2}{}}


%% \newcommand{\headerfile}[1]
%% {\nonumberlisting {{\bf 標頭檔} {\tt #1}}{#1}{}}

\newcommand{\header}[2]{\headerdetail{#1}{#2}{}}

\newcommand{\headerdetail}[3]
{{\renewcommand{\lstlistingname}{#3 \bf Header}
\lstset{basicstyle={#3 \tt}}
\lstset{label=header:#1,caption={(#1) #2}}
\lstinputlisting{#1}}}

%% Macro for prototype
%% \prototype{file}

\newcommand{\prototype}[1]{\prototypedetail{#1}{}}

%% Macro for prototype with detialed setting
%% \prototypedetail{file}{font}

\newcommand{\prototypedetail}[2]
{{\renewcommand{\lstlistingname}{#2 \bf Prototype}
\lstset{basicstyle={#2 \tt}}
\lstset{frame=single,label=prototype:#1,caption={#1}}
\lstinputlisting{#1}}}

%% Macro for phrase
%% \nphrase{file}{desc}

%% \newcommand{\nphrase}[2]
%% {{\renewcommand{\lstlistingname}{\lstfontsize \bf 片語}
%% \lstset{label=phr:#1,frame=single,caption={\lstfontsize #2}}
%% \lstset{basicstyle={\lstfontsize \tt}}
%% \lstinputlisting{#1.p}}
%% }

\newcommand{\nphrase}[2]
{\nphrasedetail{#1}{#2}{}}

%% Macro for detail phrase
%% \nphrasedetail{file}{desc}{font}

\newcommand{\nphrasedetail}[3]
{{\renewcommand{\lstlistingname}{#3 \bf 片語}
\lstset{label=phr:#1,frame=single,caption={#3 #2}}
\lstset{basicstyle={#3 \tt}}
\lstinputlisting{#1.p}}}



\newcommand{\inputfile}[1]
{\nonumberlisting {{\bf 輸入檔} {\tt #1}}{#1}{}}

\newcommand{\outputfile}[1]
{\nonumberlisting {{\bf 輸出檔} {\tt #1}}{#1}{}}

% Macro for listings that are labeled.
% [1] is the filename, [2] is the description

\newcommand{\commandline}[1]
{\lstset{basicstyle={\tt}}
\nonumberlisting{\bf 命令列}{#1.bat}{}}

% {\boxedlisting{\bf 命令列}{#1.bat}}

% Macro for listings that are labeled.
% [1] is the filename, [2] is the description

\newcommand{\out}[2]
{{\renewcommand{\lstlistingname}{\bf 執行結果}
\lstset{label=out:#1,caption=#2}
\lstinputlisting{#1.out}}}






\begin{document}

\title{Intermediate OpenMP}

\author{Pangfeng Liu \\ National Taiwan University}

\begin{frame}
  \titlepage
\end{frame}

\section{Schedule Clause}

\begin{frame}
  \frametitle{Schedule Clause} 
  \begin{itemize}
    \item A schedule clause specifies a scheduling policy for
      distributing loop iterations to threads for a {\tt parallel for}
      pragma.
  \end{itemize}
  \prototype{schedule.h}{}
\end{frame}

\begin{frame}
  \frametitle{Schedule Clause} 
  \begin{description} 
  \item [policy] The scheduling policy for distributing iterations to
    threads.
  \item [chunk] The basic unit number of iterations for scheduling.
  \end{description}
\end{frame}

\begin{frame}
  \frametitle{Scheduling Policy} OpenMP has the following scheduling
  policies for a {\tt parallel for} pragma.
  \begin{description}
  \item [static] A static round-robin policy.
  \item [dynamic] When a thread finishes it asks for more, and the
    chunk size is fixed.
  \item [guided] When a thread finishes it asks for more, and the
    chunk size decreases exponentially, and will not be smaller than
    the chunk size.
  \item [runtime] The policy is determined by an environment variable
    {\tt OMP\_SCHEDULE} or calling a function {\tt omp\_set\_policy}.
  \item [auto] This policy selects a scheduling automatically.
  \end{description}
\end{frame}

\begin{frame}
  \frametitle{Chunk Size}
  \begin{itemize}
  \item The chunk corresponds to the {\em granularity} in the
    introductory part of this course.
  \item If the granularity is large, the scheduling overhead becomes
    smaller, but it is more likely to have uneven workload distribution.
  \end{itemize}
\end{frame}

\begin{frame}
\frametitle{Dicsussion}
\begin{itemize}
\item Find out the default policy and chunk size based on the
  observation fromt the previous ``Basic OpenMP Programming'' lecture.
\end{itemize}
\end{frame}


\subsection{Static Policy}

\begin{frame}
  \frametitle{Static Policy}
  \begin{itemize}
  \item The static policy distributes the iterations in a round-robin
    fashion.
  \item This is the default policy.
  \item A thread is given a chunk size of iterations, then the next
    threads is given the same amount of iterations.
  \item Repeat this process until all iterations are distributed.
  \end{itemize}
\end{frame}

\begin{frame}
  \frametitle{Static Policy Example}
  \begin{itemize}
  \item In the following example we distribute $n$ iterations among $t$
    threads under {\tt schedule(static, 4)} policy.
  \item The $n$ and $t$ are given as command line arguments.
  \item We put a {\tt sleep(i)} to simulate the amount of work, where
    {\tt i} is the loop index.
  \item We use a variable {\tt elapsedTime} to keep track (roughly)
    the amount of time each thread runs.
  \end{itemize}
\end{frame}

\programlistingtwoslides{for-static-chunk-4.c}{Set chunk size to 4}{\footnotesize}{header}{main}{end}

\begin{frame}
  \frametitle{Demonstration}
  \begin{itemize}
  \item Run the {\tt for-static-chunk-4-omp} program with 16 iterations
    on 4 threads.
  \end{itemize}
\end{frame}

\begin{frame}
  \frametitle{Discussion}
  \begin{itemize}
  \item Describe how the iterations are distributed among the threads.
  \item Is the work evenly distributed? Observe the elapsedTime and give
    your conclusion.
  \item How to improve the workload distribution?
  \end{itemize}
\end{frame}




\begin{frame}
  \frametitle{Static Policy Example}
  \begin{itemize}
  \item In the previous example the threads with higher indices do
    more work, and the workload is not balanced among threads.
  \item One simple way to reduce work imbalance is to reduce the chunk
    size, so that the heavy iterations are more evenly distributed.
  \item We try chunk size 1 and observe the results.
  \end{itemize}
\end{frame}

\begin{frame}
  \programlistingfirst{for-static-chunk-1.c}{Set chunk size to 1}{\footnotesize}{loop}{loopend}
\end{frame}

\begin{frame}
  \frametitle{Demonstration}
  \begin{itemize}
  \item Run the {\tt for-static-chunk-1-omp} program with 16 iterations
    on 4 threads.
  \end{itemize}
\end{frame}

\begin{frame}
  \frametitle{Discussion}
  \begin{itemize}
  \item Describe how the iterations are distributed among the threads.
  \item Is the work more evenly distributed than the previous example?
    Observe the elapsedTime and give your conclusion.
  \item How to further improve the workload distribution?
  \end{itemize}
\end{frame}

\subsection{Dynamic Policy}

\begin{frame}
  \frametitle{Workload Imbalance}
  \begin{itemize}
  \item In the previous example the threads with higher indices do
    more work, and the workload is not balanced among threads.
  \item Even though we reduce the chunk size to 1, the last thread still
    has a much heavier workload than the first one.
  \end{itemize}
\end{frame}

\begin{frame}
  \frametitle{Dynamic Policy}
  \begin{itemize}
  \item The dynamic policy dictates that when a thread finishes its
    work it asks for more, and the chunk size is fixed.
  \item The work is in a {\em work pool} -- the work is put into a
    pool and any idle worker can take work from it.
  \end{itemize}
\end{frame}

\begin{frame}
  \programlistingfirst{for-dynamic-chunk-1.c}{Set scheduling to dynamic}{\footnotesize}{loop}{loopend}
\end{frame}

\begin{frame}
  \frametitle{Demonstration}
  \begin{itemize}
  \item Run the {\tt for-dynamic-chunk-1-omp} program with 16 iterations
    on 4 threads.
  \end{itemize}
\end{frame}

\begin{frame}
  \frametitle{Discussion}
  \begin{itemize}
  \item Describe how the iterations are distributed among the threads.
  \item Is the work more evenly distributed than the previous example?
    Observe the elapsedTime and give your conclusion.
  \item How to further improve the workload distribution?
  \end{itemize}
\end{frame}


\begin{frame}
  \frametitle{Workload Imbalance}
  \begin{itemize}
  \item In the previous example we run the iterations in ascending
    workload order, that is, we run the heaviest workload {\em last}.
  \item That means even if the workload is very balanced all the way to
    the end, the last work may cause all threads to wait for the last
    thread to finish.
  \end{itemize}
\end{frame}

\begin{frame}
  \frametitle{Descending Workload Order}
  \begin{itemize}
  \item In order to avoid this problem we will run the iterations in
    {\em descending} workload order, so that the dynamic policy might
    have a better change to balance the workload.
  \item We can do this because this is a {\em parallel for} -- the
    semantic explicitly demands that all iterations are independent.
  \end{itemize}
\end{frame}

\begin{frame}
  \programlistingfirst{for-dynamic-chunk-1-reverse.c}{Descending workload order}{\footnotesize}{loop}{loopend}
\end{frame}

\begin{frame}
  \frametitle{Demonstration}
  \begin{itemize}
  \item Run the {\tt for-dynamic-chunk-1-reverse-omp} program with 16
    iterations on 4 threads.
  \end{itemize}
\end{frame}

\begin{frame}
  \frametitle{Discussion}
  \begin{itemize}
  \item Describe how the iterations are distributed among the threads.
  \item Is the work more evenly distributed than the previous example?
    Observe the elapsedTime and give your conclusion.
  \end{itemize}
\end{frame}

\subsection{Guided Policy}


\begin{frame}
  \frametitle{Guided Policy}
  \begin{itemize}
  \item The {\em guided policy} dictates that when a thread finishes
    its work it asks for more, and the chunk size {\em decreases}
    exponentially, and will not be smaller than the chunk size.
  \end{itemize}
\end{frame}

\begin{frame}
  \frametitle{Rationals}
  \begin{itemize}
  \item Initial large chunk size can reduce the overheads in scheduling.
  \item Later smaller chunk size can reduce the workload imbalance.
  \item In a sense this is similar to the descending workload order --
    the larger work is given out first because of large chunk size, and
    smaller work is given out later.
  \end{itemize}
\end{frame}


\begin{frame}
  \frametitle{Comparison}
  \begin{itemize}
  \item We compare two policies -- (dynamic, 1) and (guided, 1) on
    ascending workload order.
  \item The idea is that by combining the small workload at the
    beginning, we will have a better distribution.
  \end{itemize}
\end{frame}

\begin{frame}
  \programlistingfirst{for-dynamic-chunk-1.c}{Dynamic policy for ascending workload order}{\footnotesize}{loop}{loopend}
\end{frame}

\begin{frame}
  \programlistingfirst{for-guided-chunk-1.c}{Guided policy for ascending workload order}{\footnotesize}{loop}{loopend}
\end{frame}

\begin{frame}
  \frametitle{Demonstration}
  \begin{itemize}
  \item Run the {\tt for-dynamic-chunk-1-omp} program with 32
    iterations on 4 threads.
  \item Run the {\tt for-guided-chunk-1-omp} program with 32
    iterations on 4 threads.
  \end{itemize}
\end{frame}

\begin{frame}
  \frametitle{Discussion}
  \begin{itemize}
  \item Describe how the iterations are distributed among the threads.
  \item Is the work more evenly distributed than the previous example?
    Observe the elapsedTime and give your conclusion.
  \end{itemize}
\end{frame}


\begin{frame}
  \frametitle{Worsen the Case}
  \begin{itemize}
  \item To make the matter worse, we decide to run the guided policy on
    a descending workload with chunk size 1.
  \item We compare two policies -- (dynamic, 1) and (guided, 1) on
    ascending workload order.
  \end{itemize}
\end{frame}

\begin{frame}
  \programlistingfirst{for-dynamic-chunk-1-reverse.c}{Dynamic policy for descending workload order}{\footnotesize}{loop}{loopend}
\end{frame}

\begin{frame}
  \programlistingfirst{for-guided-chunk-4-reverse.c}{Guided policy for descending workload order}{\footnotesize}{loop}{loopend}
\end{frame}

\begin{frame}
  \frametitle{Demonstration}
  \begin{itemize}
  \item Run the {\tt for-dynamic-chunk-1-reverse-omp} program with 32
    iterations on 4 threads.
  \item Run the {\tt for-guided-chunk-1-reverse-omp} program with 32
    iterations on 4 threads.
  \end{itemize}
\end{frame}

\begin{frame}
  \frametitle{Discussion}
  \begin{itemize}
  \item Describe how the iterations are distributed among the threads.
  \item Is the work more evenly distributed than the previous example?
    Observe the elapsedTime and give your conclusion.
  \end{itemize}
\end{frame}

\subsection{Runtime Policy}

\begin{frame}
  \frametitle{Runtime Policy}
  \begin{itemize}
  \item Call the function {\tt omp\_set\_schedule} to determine scheduling
    policy.
  \item We try static, dynamic, guided and auto. 
  \end{itemize}
\end{frame}

\begin{frame}
  \frametitle{Flexibility}
  \begin{itemize}
  \item We will read the policy as an integer as the fourth command line
    argument.
  \item We will read the chunk size as an integer as the fifth command
    line argument.
  \end{itemize}
\end{frame}

\begin{frame}
  \prototypedetail{omp_set_schedule.h}{\footnotesize}
  \begin{description}
  \item [kind] for the scheduling policy.
  \item [modifier] for the chunk size.
  \end{description}
\end{frame}

\begin{frame}
  \frametitle{}
  \begin{itemize}
  \item omp\_sched\_static (integer 1) for static
  \item omp\_sched\_dynamic (integer 2) for dynamic
  \item omp\_sched\_guided (integer 3) for guided
  \item omp\_sched\_auto (integer 4) for auto
  \end{itemize}
\end{frame}

\programlistingtwoslides{for-runtime.c}{Runtime policy for ascending workload order}{\footnotesize}{main}{loop}{end}


\begin{frame}
  \frametitle{Demonstration}
  \begin{itemize}
  \item Run the {\tt for-runtime-omp} program with 16 iterations on 4
    threads, under various combinations of policy and chunk sizes.
  \end{itemize}
\end{frame}

\begin{frame}
  \frametitle{Discussion}
  \begin{itemize}
  \item Describe how the iterations are distributed among the threads
    when the policy is auto.
  \item Which policy distributes workload more evenly than others?
    Observe the elapsedTime and give your conclusion.
  \end{itemize}
\end{frame}

\section{Measure Time}

\begin{frame}
  \prototype{omp_get_wtime.h}
  \begin{itemize}
  \item Return the current time.
  \item Can measure the length of a window by two consecutive calls.
  \end{itemize}
\end{frame}

\section{Parallel Sections}

\begin{frame}
  \frametitle{Parallel Sections}
  \begin{itemize}
  \item Want to run a set of computations in parallel, e.g.,
    initializing two matrices {\tt a} and {\tt b}.
  \item Can have a thread to initialize {\tt a} and another thread for
    {\tt b}.
  \item Use parallel sections pragma.
  \end{itemize}
\end{frame}

\begin{frame}
  \frametitle{Parallel Sections}
  \prototype{omp_parallel_section.h}
\end{frame}

\begin{frame}
  \frametitle{Naive Parallelization}
  \begin{itemize}
  \item A parallel program initializes two matrices {\tt a} and {\tt
    b} using {\tt \#pragma omp parallel} sections.
  \item The sizes of both matrices are 8192 by 8192.
  \item We use {\tt omp\_get\_wtime} to measure time.
  \item Finally we check for correctness of the program by
    assertions.
  \end{itemize}
\end{frame}

\programlistingthreeslides{2loops.c}{}{\footnotesize}{header}{sections}{sectionsend}{end}

\begin{frame}
  \frametitle{Demonstration}
  \begin{itemize}
  \item Run the {\tt 2loops-omp} program.
  \end{itemize}
\end{frame}

\begin{frame}
  \frametitle{Discussion}
  \begin{itemize}
  \item Does this parallel program correctly initialize the matrices?
    What could be the reason if it does not?
  \item Can you conclude which thread will run which section?
  \end{itemize}
\end{frame}

\begin{frame}
  \frametitle{Index Variables}
  \begin{itemize}
  \item The problem with the previous program is that two threads
    running two sections share the index variables {\tt i} and {\tt
      j}.
  \item There are various ways to fix the problem, and we will use
    the most intuitive one -- by declaring the index variables
    within the initialization part of for loop, as c99-style syntax.
  \end{itemize}
\end{frame}

\programlistingthreeslides{2loops-stdc99.c}{}{\footnotesize}{header}{sections}{sectionsend}{end}

\begin{frame}
  \frametitle{Demonstration}
  \begin{itemize}
  \item Run the {\tt 2loops-stdc99-omp} program.
  \end{itemize}
\end{frame}

\begin{frame}
  \frametitle{Discussion}
  \begin{itemize}
  \item Does this parallel program correctly initialize the matrices?
    How do you know?
  \item Can you conclude which thread will run which section?
  \end{itemize}
\end{frame}

\section{Parallel Directive}

\begin{frame}
  \frametitle{{\tt parallel} Revisited}
  \begin{itemize}
  \item We have discussed {\tt parallel sections} and {\tt parallel for}
    directives, but in two separate contexts.
  \item Now we want to combine the idea and describe the true meaning of
    {\tt parallel}, which is to ``run the following statement with
    multiple threads''.
  \item If then we encounter a {\tt sections}, then we have a {\tt
    parallel sections}.
  \item If then we encounter a {\tt for}, then we have a {\tt parallel
    for}.
  \end{itemize}
\end{frame}

\begin{frame}
  \frametitle{\tt parallel}
  \prototype{omp_parallel.h}
  \begin{itemize}
  \item Run the following statement with multiple threads.
  \end{itemize}
\end{frame}

\begin{frame}
  \frametitle{{\tt parallel} Examples}
  \programlisting{parallel-sections.c}{An Example}{\scriptsize}
\end{frame}

\begin{frame}
  \frametitle{{\tt parallel} Examples}
  \programlisting{parallel-sections-wrong.c}{A Wrong Example}{\footnotesize}
\end{frame}

\begin{frame}
  \frametitle{Compile Error}
  \begin{itemize}
  \item The previous code will not compile because a {\tt section} must
    be within a {\tt sections}.
  \item Unfortunately the effect of an {\tt sections} is only for the
    following statement, so we need to use compound statement to include
    {\em all} {\em section} statements.
  \end{itemize}
\end{frame}

\begin{frame}
  \frametitle{{\tt parallel} Examples}
  \programlisting{parallel-1for.c}{An Examples}{\footnotesize}
\end{frame}

\begin{frame}
  \frametitle{Multiply and Divide}
  \begin{itemize}
  \item These two are equivalent.
  \item One can think of the {\tt parallel} is to spawn multiple threads, and the {\tt for} is to divide work.
  \end{itemize}
\end{frame}

\begin{frame}
  \frametitle{{\tt parallel} Examples}
  \programlisting{parallel-2for.c}{An Examples}{\footnotesize}
\end{frame}

\begin{frame}
  \frametitle{Multiply and Divide}
  \begin{itemize}
  \item These two are also equivalent.
  \item One can think of the {\tt parallel} is to spawn multiple
    threads, and the first {\tt for} is to divide the work, and the
    second {\tt for} is to divide the work {\em again}.
  \end{itemize}
\end{frame}

\begin{frame}
  \frametitle{{\tt parallel} Examples}
  \programlisting{parallel-2for-wrong.c}{An Wrong Examples}{\footnotesize}
\end{frame}

\begin{frame}
  \frametitle{Multiply and Divide}
  \begin{itemize}
  \item These two are {\em not} equivalent.
  \item One can think of the {\tt parallel} is to spawn multiple
    threads, and the first {\tt for} is to divide the work.
  \item Since the {\tt parallel} covers only one statement, now we
    switch back to single thread.
  \item The second {\tt for} has no multiple threads to divide the work!
  \item Unlike the previous {\tt sections} problem, this will {\em not}
    cause compile error and will simply degrade performance.
  \end{itemize}
\end{frame}


\begin{frame}
  \frametitle{Both Section and For}
  \begin{itemize}
  \item Want to use both sections and for in {\tt 2loops-std99.c}
  \item We already have a thread initializing {\tt a} and another
    thread initializing {\tt b}.
  \item Now we use {\tt parallel for} to parallel the asserts at the
    end.
  \end{itemize}
\end{frame}

\programlistingtwoslides{2loops-stdc99-both.c}{}{\scriptsize}{parallel sections}{parallelfor}{end}

\begin{frame}
\frametitle{Structure}
\begin{itemize}
\item {\tt parallel}
\begin{itemize}
\item {\tt sections}
\begin{itemize}
\item {\tt section}
\item {\tt section}
\end{itemize}
\end{itemize}
\item {\tt parallel}
\begin{itemize}
\item {\tt for}
\item {\tt for}
\end{itemize}
\end{itemize}
\end{frame}

\begin{frame}
\frametitle{Meaning}
\begin{itemize}
\item {\tt parallel} Go multithreading
\begin{itemize} 
\item {\tt sections} Divide the work
\begin{itemize}
\item {\tt section} 
\item {\tt section}
\end{itemize}
\end{itemize}
\item {\tt parallel} Go multithreading
\begin{itemize}
\item {\tt for}  Divide the work
\item {\tt for}  Divide the work
\end{itemize}
\end{itemize}
\end{frame}

\begin{frame}
  \frametitle{Demonstration}
  \begin{itemize}
  \item Run the {\tt 2loops-stdc99-both-omp} program.
  \end{itemize}
\end{frame}

\begin{frame}
  \frametitle{Discussion}
  \begin{itemize}
  \item Observe the thread indices reported by {\tt
    2loops-stdc99-both-omp} and confirm that multi-threading is
    active.
  \item Can you conclude which thread will run which section?
  \item Can you conclude which thread will run which loop?
  \end{itemize}
\end{frame}


\begin{frame}
  \frametitle{One Statement after {\tt parallel}}
  \begin{itemize}
  \item What will happen if we do {\em not} place everything as a
    compound statement after {\tt parallel}?
  \end{itemize}
\end{frame}

\programlistingtwoslides{2loops-stdc99-both-wrong.c}{}{\scriptsize}{parallelsections}{parallelfor}{end}


\begin{frame}
\frametitle{Meaning}
\begin{itemize}
\item {\tt parallel} Go multithreading
\begin{itemize} 
\item {\tt sections} Divide the work
\begin{itemize}
\item {\tt section} 
\item {\tt section}
\end{itemize}
\end{itemize}
\item {\tt parallel} Go multithreading
\begin{itemize}
\item {\tt for}  Divide the work
\end{itemize}
\item {\tt for}  Divide the work???
\end{itemize}
\end{frame}

\begin{frame}
\frametitle{Problem}
\begin{itemize}
\item Removing the compound statement of the first {\tt parallel} will
  not cause any problem because the {\tt sections} must use a compound
  statement to include two {\tt section}.  Otherwise it will not
  compile.
\item Removing the compound statement of the first {\tt parallel}
  cause performance problem because the second {\tt parallel} does not
  cover the second {\tt for}, so it will {\em not} go multithreading.
\end{itemize}
\end{frame}


\begin{frame}
  \frametitle{Demonstration}
  \begin{itemize}
  \item Run the {\tt 2loops-stdc99-both-wrong-omp} program.
  \end{itemize}
\end{frame}

\begin{frame}
  \frametitle{Discussion}
  \begin{itemize}
  \item Observe the thread indices reported by {\tt
    2loops-stdc99-both-wrong-omp} and confirm that multi-threading is
    {\em not active} in the last loop.
  \item Can you conclude which thread will run which section?
  \item Can you conclude which thread will run which loop?
  \end{itemize}
\end{frame}


\begin{frame}
  \frametitle{One {\tt parallel}}
  \begin{itemize}
  \item Now we put everything into a single {\tt parallel}.
  \end{itemize}
\end{frame}

\programlistingtwoslides{2loops-stdc99-combined.c}{}{\scriptsize}{parallelsections}{parallelfor}{end}


\begin{frame}
\frametitle{Structure}
\begin{itemize}
\item {\tt parallel} Go multithreading
  \begin{itemize}
  \item {\tt sections} Divide the work
    \begin{itemize}
    \item {\tt section}
    \item {\tt section}
    \end{itemize}
  \item {\tt for} Divide the work
  \item {\tt for} Divide the work
  \end{itemize}
\end{itemize}
\end{frame}

\begin{frame}
\frametitle{Efficiency}
\begin{itemize}
\item The program will go multi-threading only {\em once}.
\item This removes the overheads in creating and destroying the
  threads.
\end{itemize}
\end{frame}

\begin{frame}
  \frametitle{Demonstration}
  \begin{itemize}
  \item Run the {\tt 2loops-stdc99-combine-omp} program.
  \end{itemize}
\end{frame}

\begin{frame}
  \frametitle{Discussion}
  \begin{itemize}
  \item Observe the thread indices reported by {\tt
    2loops-stdc99-combined-omp} and confirm that multi-threading is
    active everywhere.
  \item Can you conclude which thread will run which section?
  \item Can you conclude which thread will run which loop?
  \end{itemize}
\end{frame}


%% \begin{frame}
%%   \frametitle{Prime Number Counting}
%%   \begin{itemize}
%%   \item We want to count the number of prime numbers.
%%   \item We start with an array of numbers, and assuming that every
%%     number is a prime number.
%%   \item We start with the smallest prime number in the array, and mark
%%     {\em every} multiple of it as {\em composite} (non-prime number).
%%   \item We repeat this process until no new prime numbers are found.
%%   \end{itemize}
%% \end{frame}

%% \programlistingthreeslides{prime.c}{}{\small}{header}{main}{count}{end}

%% \begin{frame}
%% \frametitle{Variables}
%% \begin{itemize}
%% \item Array {\tt notPrime} keeps track of the status of a number.  If
%%   we know a number {\tt i} is {\em not} prime, we set {\tt
%%     notPrime[i]} to 1.
%% \item The prime numbers will be kept in another array {\tt primes}.
%% \item The range to be tested ({\tt n}) is given as a command line argument.
%% \end{itemize}
%% \end{frame}

%% \begin{frame}
%% \frametitle{Try All Possibilities}
%% \begin{itemize}
%% \item Try all numbers from 2 to $\sqrt{n}$.
%% \item If i is a prime number, mark the all multiple of {\tt i} as {\em
%%   not} prime.
%% \item It is obvious that the first for loop {\em cannot} be
%%   parallelized because of dependency, so we parallelize the second for
%%   loop.
%% \end{itemize}
%% \end{frame}

%% \begin{frame}
%% \frametitle{Reduction}
%% \begin{itemize}
%% \item The number of prime number can be obtained by counting the
%%   number of zeros in array {\tt notPrime}.
%% \item We use a reduction on the variable {\tt nPrime} to simplify the
%%   process.
%% \end{itemize}
%% \end{frame}

%% \begin{frame}
%%   \frametitle{Demonstration}
%%   \begin{itemize}
%%   \item Run the {\tt prime-uni} program.
%%   \item Run the {\tt prime-omp} program.
%%   \end{itemize}
%% \end{frame}

%% \begin{frame}
%%   \frametitle{Discussion}
%%   \begin{itemize}
%%   \item Compute the speedup of the previous prime counting program.
%%   \item Is there any optimization that can improve the performance?
%%   \end{itemize}
%% \end{frame}

%% \begin{frame}
%% \frametitle{Efficiency}
%% \begin{itemize}
%% \item The previous program parallelized the inner for loop only.
%% \item The previous program will go through multi-threading every time a
%%   prime number is found.
%% \item This incurs overheads of creating and destroying the threads.
%% \end{itemize}
%% \end{frame}

%% \begin{frame}
%% \frametitle{Spawn and Join}
%% \centerline{\pgfimage[width=1.0\textwidth]{threads-many-spawn-join.pdf}}
%% \begin{itemize}
%% \item Single thread for the outer loop.
%% \item Multi-thread for the inner loop.
%% \end{itemize}
%% \end{frame}


%% \begin{frame}
%% \frametitle{{\tt parallel} Once}
%% \begin{itemize}
%% \item We would like to avoid the overheads in creating and destroying
%%   threads, so we put a {\tt parallel} in front of the first for loop.
%% \item The {\em entire} two level loop is run by all threads.
%% \item Then we share the workload of the second for loop to improve
%%   performance,  since  most work is done in the second loop.
%% \end{itemize}
%% \end{frame}

%% \begin{frame}
%% \programlistingfirst{prime-inner.c}{}{\small}{main}{count}
%% \end{frame}

%% \begin{frame}
%% \frametitle{Efficient}
%% \begin{itemize}
%% \item All threads run the first loop, and share the second loop.
%% \item It is OK for all threads to run the first for loop
%%   simultaneously, since they will synchronize at every second loop.
%% \item Index variables {\tt i} and {\tt j} are private.
%% \end{itemize}
%% \end{frame}

%% \begin{frame}
%% \frametitle{Spawn and Join}
%% \centerline{\pgfimage[width=1.0\textwidth]{threads-one-spawn-join.pdf}}
%% \begin{itemize}
%% \item Multi-thread for both outer and inner loops.
%% \item The work in the outer loop is duplicated, but it is more
%%   efficient than creating and joining threads.
%% \end{itemize}
%% \end{frame}



%% \begin{frame}
%%   \frametitle{Demonstration}
%%   \begin{itemize}
%%   \item Run the {\tt prime-uni} program.
%%   \item Run the {\tt prime-omp} program.
%%   \item Run the {\tt prime-inner-omp} program.
%%   \end{itemize}
%% \end{frame}

%% \begin{frame}
%%   \frametitle{Discussion}
%%   \begin{itemize}
%%   \item Compare the execution time of all three prime counting
%%     programs.
%%   \item Is there any further optimization that can improve the
%%     performance?
%%   \end{itemize}
%% \end{frame}




%% \begin{frame}
%%   \frametitle{Traveling Salesman} Modify your $N$-queen program into
%%   a sequential traveling salesman program and submit it.
%%   \begin{itemize}
%%   \item The distance between two points are ${dx}^2 + {dy}^2$ (no
%%     sqrt!).
%%   \item The traveling salesman does not need to go back to the city
%%     where he started, so the solution is a {\em permutation} of all
%%     cities.
%%   \item Read the input form stdin.  The first line is the number of
%%     points and each line after has the $x$ and $y$ coordinates of a
%%     city.
%%   \item The output is the minimum total distance.
%%   \end{itemize}
%% \end{frame}

%% \begin{frame}
%%   \frametitle{Cuts} 
%%   \begin{itemize}
%%   \item Add cuts into your traveling salesman program and submit it.
%%   \item What concept do you need in order to maintain the minimum
%%     solution so far?
%%   \end{itemize}
%% \end{frame}

%% \begin{frame}
%%   \frametitle{Parallel Traveling Salesman} Now parallelize your
%%   traveling salesman program.
%%   \begin{itemize}
%%   \item What concept do you need in order to maintain the minimum
%%     solution so far? 
%%   \item Do you need to protect this variable since now we may have
%%     multiple thread accessing it? 
%%   \end{itemize}
%% \end{frame}

%% \begin{frame}
%%   \frametitle{Timing} 
%%   \begin{itemize}
%%   \item Use {\tt omp\_get\_wtime()} to measure the time for each thread. 
%%   \item Do you think it is necessary to load balance your code for
%%     different threads in this case?
%%   \item Time the parallel program with {\tt guided} and {\tt
%%     dynamic} policies.
%%   \end{itemize}
%% \end{frame}

%% \begin{frame}
%%   \frametitle{Game of Life} Submit a game of life program.
%%   \begin{itemize}
%%   \item A dead cell with exactly {\em three} live neighbors becomes
%%     a live cell. 
%%   \item A live cell with two or three live neighbors stays
%%     live, otherwise it become dead. 
%%   \end{itemize}
%% \end{frame}

%% \begin{frame}
%%   \frametitle{Input and Output}
%%   \begin{itemize}
%%   \item The first line of the input is the size of the board
%%     $N$. 
%%   \item The next $N$ lines has $N$ integers each.  A $0$ is a dead
%%     cell and and a $1$ is a live cell.  $N \leq 3000$.
%%   \item Output the result after each iteration.
%%   \item Print the answer just like in the input.
%%   \end{itemize}
%% \end{frame}

%% \begin{frame}
%%   \frametitle{Parallel Game of Life} Submit a parallel version of
%%   life. 
%%   \begin{itemize}
%%   \item Compare the performance with the sequential version. 
%%   \item You may need to make the array {\em very} big in order to
%%     see the performance improvement.
%%   \item Do you think the dynamic or guided scheduling will help
%%     performance?  Justify your answer by both reasoning and
%%     experiments.
%%   \end{itemize}
%% \end{frame}


%% \begin{frame}
%%   \frametitle{Timing} 
%%   \prototype{omp_get_wtime}{Measure the elapsed time in second}
%%   \begin{itemize}
%%   \item You need to make two calls and calculate the difference between
%%     the return values.
%%   \end{itemize}
%% \end{frame}

%% \begin{frame}
%%   \frametitle{Compute Square Roots}
%%   Now write a program that computes the square root from 0 to
%%   999999999 and assign the value to a variable v, and observe its
%%   timing with {\tt omp\_get\_wtime}.
%% \end{frame}

%% \begin{frame}
%%   \frametitle{Timing Comparison}
%%   Now write two parallel versions of the previous program.  Both
%%   programs use {\tt parallel for} pragma.  Compare the timing of these
%%   two programs.
%%   \begin{itemize}
%%   \item Declare {\tt v} as private.
%%   \item Declare {\tt v} as shared.
%%   \end{itemize}
%% \end{frame}

%% \begin{frame}
%%   \frametitle{firstprivate and lastprivate}
%%   \prototype{firstlastprivate}{{\tt firstprivate} and {\tt lastprivate}}
%%   \begin{itemize}
%%   \item A {\tt firstprivate} clause causes the private variables of all
%%     threads to inherit an initial value from the master thread.
%%   \item A {\tt lastprivate} clause causes the master thread to receive
%%     the value of a variable from the last thread.
%%   \end{itemize}
%% \end{frame}

%% \begin{frame}
%%   \frametitle{An Example}
%%   \programlistingoption{firstlast}{{\tt firstprivate} and {\tt lastprivate} example}{basicstyle={\tt \small}}
%% \end{frame}

%% \begin{frame}
%%   \frametitle{Last Thread}
%%   \begin{itemize}
%%   \item Submit the private {\tt v} version you wrote previously. 
%%   \item For ease of grading we only print the sum of square roots from
%%     the last thread.  The output format is {\tt \%lf}.
%%   \item We can initialize the global {\tt v} to 0.0, and ``carry'' the
%%     initialization to every thread by {\tt firstprivate}
%%   \item The master thread can receive the final result of the last
%%     thread by {\tt lastprivate}.
%%   \end{itemize}
%% \end{frame}

%% \begin{frame}
%%   \frametitle{Compute $\pi$} 
%%   \begin{itemize}
%%   \item Write a sequential program to calculate $\pi$ by integrating
%%     $f(x) = {4 \over {1 + x^2}}$, where $x$ is from 0 to 1.  
%%   \item Divide the interval into $N$ pieces, and assume the area in each
%%     interval is a trapezoid, then sum the area of these $N$ trapezoids
%%     into a variable {\tt area}.  Note that in each interval you only
%%     need to compute $4 \over {1 + x^2}$ once (where is it?).
%%   \item We assume $N$ to be 1000000. The output format is {\tt \%.10lf}.
%%   \end{itemize}
%% \end{frame}

%% \begin{frame}
%%   \programlisting{pi-loop}{Compute $\pi$}
%% \end{frame}
%% \begin{frame}
%%   \frametitle{Parallel Version} 
%%   \begin{itemize}
%%   \item Now write a parallel version by only adding {\tt parallel for}
%%     pragma.  Remember to declare {\tt x} as private.
%%   \item Make sure that the answer is {\em incorrect}.
%%   \end{itemize}
%% \end{frame}

%% \begin{frame}
%%   \frametitle{Parallel Version} 
%%   \begin{itemize}
%%   \item Now write a parallel version by only adding {\tt parallel for}
%%     pragma.
%%   \item Make sure that the answer is {\em incorrect}.
%%   \end{itemize}
%% \end{frame}

%% \begin{frame}
%%   \frametitle{Critical pragma} 
%%   \prototype{critical}{Critical pragma}
%%   \begin{itemize}
%%   \item When a thread enters the critical section, the others could
%%     not.
%%   \item The {\tt area} is {\em protected} by a critical section.
%%   \end{itemize}
%% \end{frame}

%% \begin{frame}
%%   \frametitle{Critical pragma} Now write two parallel versions using
%%   critical section 
%%   \begin{itemize}
%%   \item The critical section covers the entire loop body to avoid
%%     race condition.
%%   \item The critical section covers the addition to the area to
%%     avoid race condition.
%%   \end{itemize}
%% \end{frame}

%% \begin{frame}
%%   \frametitle{Critical pragma} 
%%   \begin{itemize}
%%   \item Compare the timing between the two parallel versions you
%%     just wrote.  Which one is more efficient?  Why?
%%   \item Submit the one that is more efficient?
%%   \end{itemize}
%% \end{frame}


%% \begin{frame}
%%   \frametitle{Critical pragma} Now submit a parallel $\pi$ program
%%   using reduction and compare the timing among the three parallel
%%   versions.
%%   \begin{itemize} 
%%   \item The big critical section
%%   \item The small critical section
%%   \item The reduction
%%   \end{itemize}
%% \end{frame}

%% \begin{frame}
%%   \frametitle{Critical pragma} 
%%   \begin{itemize} 
%%   \item Time the implementation using reduction with different
%%     number of threads.
%%   \item Report the speedup you observe.
%%   \end{itemize}
%% \end{frame}

%% \begin{frame}
%%   \frametitle{Critical pragma} 
%%   \begin{itemize} 
%%   \item Time the implementation using reduction with different
%%     number of threads.
%%   \item Report the speedup you observe.
%%   \end{itemize}
%% \end{frame}

%% \begin{frame}
%%   \frametitle{Critical pragma} 
%%   \begin{itemize} 
%%   \item Time the implementation using reduction with different
%%     number of threads.
%%   \item Report the speedup you observe.
%%   \end{itemize}
%% \end{frame}

%\end{CJK}
\end{document}
