\documentclass{beamer}
\usetheme{Warsaw}

\usepackage{color}
\usepackage{listings}
\usepackage{url}
\usepackage{booktabs}
\usepackage{pgf}
\usepackage{multicol}
\usepackage{algorithm}
\usepackage{algorithmic}


% global setting

\newcommand{\comment}[1]{}

% list environment gloabal setting

% to specify the size of fonts in lst environment

\lstset{showstringspaces=false}
\lstset{language=C}
\lstset{basicstyle=\tt}
\lstset{keywordstyle=\color{blue}}
\lstset{emphstyle=\color{red}}
\lstset{numbers=left}
\lstset{frame=single}
\lstset{rangeprefix=\/*\ }
\lstset{rangesuffix=\ *\/}
\lstset{includerangemarker=false}

%% slides related macros

\newenvironment{mytitle}[2]
{\begin{frame}{#1} \vspace{0.2\textheight}{\huge #2} \end{frame}}

%% Macro for non-numbered frames listing 
%% \nonumberlisting{title}{file}{options}

\newcommand{\nonumberlisting}[3]
{\noindent{#1}
{\lstinputlisting[#3]{#2}}}

%% Program listing macros

%% \programlisting{file}{desp}{font}
%% \programlistingfirst{file}{desp}{font}{start}{end}
%% \programlistingrest{file}{descp}{font}{start}{end}


%% List a program in one page with a font size.

\newcommand{\programlisting}[3]
{\renewcommand{\lstlistingname}{#3 \bf Example}
\lstinputlisting[basicstyle={#3 \tt},caption={#3 \bf (#1) #2}]{#1}}

%% List a program for the first page within a given range.

\newcommand{\programlistingfirst}[5]
{\renewcommand{\lstlistingname}{#3 \bf Example}
\lstinputlisting[basicstyle={#3 \tt},linerange={#4-#5},caption={#3 \bf (#1) #2}]{#1}}

%% List a program for the second page (or later).

\newcommand{\programlistingrest}[5]
{\lstinputlisting[basicstyle={#3 \tt},linerange={#4-#5}]{#1}}

%% List a program in two pages.

\newcommand{\programlistingtwoslides}[6]
{\begin{frame}                  % do not add [fragile]
\programlistingfirst{#1}{#2}{#3}{#4}{#5}
\end{frame}
\begin{frame}
\programlistingrest{#1}{#2}{#3}{#5}{#6}
\end{frame}}

%% List a program in three pages.

\newcommand{\programlistingthreeslides}[7]
{\begin{frame}                  % do not add [fragile]
\programlistingfirst{#1}{#2}{#3}{#4}{#5}
\end{frame}
\begin{frame}
\programlistingrest{#1}{#2}{#3}{#5}{#6}
\end{frame}
\begin{frame}
\programlistingrest{#1}{#2}{#3}{#6}{#7}
\end{frame}
}


%% Program input/output macros

%% \programoutput{file}
%% \programoutputfirst{file}{font}{options}
%% \programoutputrest{file}{font}{startline}{endline}
%% \programoutputtwoslides{file}{font}{start}{endl}{start2}{end2}

%% Program output macor with default setting

\newcommand{\programoutput}[1]
{\programoutputfirst{#1}{}{}}

%% Program output macor for the first page

\newcommand{\programoutputfirst}[3]
{\lstset{basicstyle={#2 \tt}}
\nonumberlisting{#2 \bf 輸出}{#1.out}{#3}}

%% Program output macor for later page

\newcommand{\programoutputrest}[4]
{\lstinputlisting[firstnumber={#3},basicstyle={\tt #2},linerange={#3-#4}]
{#1.out}}

%% Program output macor for two slides

\newcommand{\programoutputtwoslides}[6]
{\begin{frame}                  % do not add [fragile]
\programoutputfirst{#1}{#2}{linerange={#3-#4}}
\end{frame}
\begin{frame}
\programoutputrest{#1}{#2}{#5}{#6}
\end{frame}}


%% \programinput{file}
%% \programinputfirst{file}{font}{option}

%% Program input macor with default setting

\newcommand{\programinput}[1]
{\programinputfirst{#1}{}{}}

%% Program input macor for the first page

\newcommand{\programinputfirst}[3]
{\lstset{basicstyle={#2 \tt}}
\nonumberlisting{#2 \bf 輸入}{#1.in}{#3}}


%% \piotwocol{file}

\newcommand{\piotwocol}[1]{\piotwocoldetail{#1}{}{}}

\newcommand{\piotwocoldetail}[2]{
\begin{columns}[t,onlytextwidth]
\begin{column}{0.45\textwidth}
\programinputfirst{#1}{#2}{}
\end{column}
\begin{column}{0.45\textwidth}
\programoutputfirst{#1}{#2}{}
\end{column}
\end{columns}}

%% \piotworow{file}

\newcommand{\piotworow}[1]
{\programinput{#1}
\programoutput{#1}}

\newcommand{\piotworowdetail}[2]
{\programinputfirst{#1}{#2}{}
\programoutputfirst{#1}{#2}{}}


%% \newcommand{\headerfile}[1]
%% {\nonumberlisting {{\bf 標頭檔} {\tt #1}}{#1}{}}

\newcommand{\header}[2]{\headerdetail{#1}{#2}{}}

\newcommand{\headerdetail}[3]
{{\renewcommand{\lstlistingname}{#3 \bf Header}
\lstset{basicstyle={#3 \tt}}
\lstset{label=header:#1,caption={(#1) #2}}
\lstinputlisting{#1}}}

%% Macro for prototype
%% \prototype{file}

\newcommand{\prototype}[1]{\prototypedetail{#1}{}}

%% Macro for prototype with detialed setting
%% \prototypedetail{file}{font}

\newcommand{\prototypedetail}[2]
{{\renewcommand{\lstlistingname}{#2 \bf Prototype}
\lstset{basicstyle={#2 \tt}}
\lstset{frame=single,label=prototype:#1,caption={#1}}
\lstinputlisting{#1}}}

%% Macro for phrase
%% \nphrase{file}{desc}

%% \newcommand{\nphrase}[2]
%% {{\renewcommand{\lstlistingname}{\lstfontsize \bf 片語}
%% \lstset{label=phr:#1,frame=single,caption={\lstfontsize #2}}
%% \lstset{basicstyle={\lstfontsize \tt}}
%% \lstinputlisting{#1.p}}
%% }

\newcommand{\nphrase}[2]
{\nphrasedetail{#1}{#2}{}}

%% Macro for detail phrase
%% \nphrasedetail{file}{desc}{font}

\newcommand{\nphrasedetail}[3]
{{\renewcommand{\lstlistingname}{#3 \bf 片語}
\lstset{label=phr:#1,frame=single,caption={#3 #2}}
\lstset{basicstyle={#3 \tt}}
\lstinputlisting{#1.p}}}



\newcommand{\inputfile}[1]
{\nonumberlisting {{\bf 輸入檔} {\tt #1}}{#1}{}}

\newcommand{\outputfile}[1]
{\nonumberlisting {{\bf 輸出檔} {\tt #1}}{#1}{}}

% Macro for listings that are labeled.
% [1] is the filename, [2] is the description

\newcommand{\commandline}[1]
{\lstset{basicstyle={\tt}}
\nonumberlisting{\bf 命令列}{#1.bat}{}}

% {\boxedlisting{\bf 命令列}{#1.bat}}

% Macro for listings that are labeled.
% [1] is the filename, [2] is the description

\newcommand{\out}[2]
{{\renewcommand{\lstlistingname}{\bf 執行結果}
\lstset{label=out:#1,caption=#2}
\lstinputlisting{#1.out}}}






\begin{document}

\title{Advanced OpenMP Programming}

\author{Pangfeng Liu \\ National Taiwan University}

\begin{frame}
  \titlepage
\end{frame}

\section{Performance}

\subsection{Private Variables}

\begin{frame}
  \begin{itemize}
  \item We will observe the timing when threads access variables at
    different locations.
  \item The intuition is that local variables are much easier to
    access than the global ones, and we need to confirm that.
  \end{itemize}
\end{frame}

\begin{frame}
  \frametitle{Compute Square Roots}
  \begin{itemize}
    \item Write a program that computes the square root from 0 to
      999999999 and assign the value to a variable {\tt v}.
    \item We first observe the timing when we place the variable {\tt
      v} at global area.
  \end{itemize}
\end{frame}

\begin{frame}
\programlisting{assign.c}{}{}
\end{frame}

\begin{frame}
  \frametitle{Demonstration}
  \begin{itemize}
  \item Run the {\tt assign-uni} program.
  \item Run the {\tt assign-omp} program.
  \end{itemize}
\end{frame}

\begin{frame}
  \frametitle{Discussion}
  \begin{itemize}
  \item Compare the execution time of {\tt assign-uni} and {\tt assign-omp}.
  \end{itemize}
\end{frame}


\begin{frame}
  \frametitle{Private Variable}
  \begin{itemize}
  \item We now declare {\tt v} as private and observe the timing.
  \end{itemize}
\end{frame}

\begin{frame}
\programlisting{assign-private.c}{}{}
\end{frame}

\begin{frame}
  \frametitle{Demonstration}
  \begin{itemize}
  \item Run the {\tt assign-private-omp} program.
  \end{itemize}
\end{frame}

\begin{frame}
  \frametitle{Discussion}
  \begin{itemize}
  \item Compare the execution time of {\tt assign-private-omp} with
    previous {\tt assign-uni} and {\tt assign-omp}.
    \item What is the reason for this performance difference?
  \end{itemize}
\end{frame}

\begin{frame}
  \frametitle{Heap Variable}
  \begin{itemize}
  \item We now put {\tt v} into the heap and observe the timing.
  \end{itemize}
\end{frame}

\begin{frame}
\programlisting{assign-heap.c}{}{}
\end{frame}

\begin{frame}
  \frametitle{Demonstration}
  \begin{itemize}
  \item Run the {\tt assign-heap-omp} program.
  \end{itemize}
\end{frame}

\begin{frame}
  \frametitle{Discussion}
  \begin{itemize}
  \item Compare the execution time of {\tt assign-heap-omp} with
    previous programs.
  \item What is the reason for this performance difference?
  \end{itemize}
\end{frame}



%% \begin{frame}
%%   \frametitle{Timing Comparison}
%%   Now write two parallel versions of the previous program.  Both
%%   programs use {\tt parallel for} pragma.  Compare the timing of these
%%   two programs.
%%   \begin{itemize}
%%   \item Declare {\tt v} as private.
%%   \item Declare {\tt v} as shared.
%%   \end{itemize}
%% \end{frame}




\subsection{Thread Creation/Destroying}

\begin{frame}
  \frametitle{Prime Number Counting}
  \begin{itemize}
  \item We want to count the number of prime numbers.
  \item We start with an array of numbers, and assuming that every
    number is a prime number.
  \item We start with the smallest prime number in the array, and mark
    {\em every} multiple of it as {\em composite} (non-prime number).
  \item We repeat this process until no new prime numbers are found.
  \end{itemize}
\end{frame}

\programlistingthreeslides{prime.c}{}{\small}{header}{main}{count}{end}

\begin{frame}
\frametitle{Variables}
\begin{itemize}
\item Array {\tt notPrime} keeps track of the status of a number.  If
  we know a number {\tt i} is {\em not} prime, we set {\tt
    notPrime[i]} to 1.
\item The prime numbers will be kept in another array {\tt primes}.
\item The range to be tested ({\tt n}) is given as a command line argument.
\end{itemize}
\end{frame}

\begin{frame}
\frametitle{Try All Possibilities}
\begin{itemize}
\item Try all numbers from 2 to $\sqrt{n}$.
\item If i is a prime number, mark the all multiple of {\tt i} as {\em
  not} prime.
\item It is obvious that the first for loop {\em cannot} be
  parallelized because of dependency, so we parallelize the second for
  loop.
\end{itemize}
\end{frame}

\begin{frame}
\frametitle{Reduction}
\begin{itemize}
\item The number of prime number can be obtained by counting the
  number of zeros in array {\tt notPrime}.
\item We use a reduction on the variable {\tt nPrime} to simplify the
  process.
\end{itemize}
\end{frame}

\begin{frame}
  \frametitle{Demonstration}
  \begin{itemize}
  \item Run the {\tt prime-uni} program.
  \item Run the {\tt prime-omp} program.
  \end{itemize}
\end{frame}

\begin{frame}
  \frametitle{Discussion}
  \begin{itemize}
  \item Compute the speedup of the previous prime counting program.
  \item Is there any optimization that can improve the performance?
  \end{itemize}
\end{frame}

\begin{frame}
\frametitle{Efficiency}
\begin{itemize}
\item The previous program parallelized the inner for loop only.
\item The previous program will go through multi-threading every time a
  prime number is found.
\item This incurs overheads of creating and destroying the threads.
\end{itemize}
\end{frame}

\begin{frame}
\frametitle{Spawn and Join}
\centerline{\pgfimage[width=1.0\textwidth]{threads-many-spawn-join.pdf}}
\begin{itemize}
\item Single thread for the outer loop.
\item Multi-thread for the inner loop.
\end{itemize}
\end{frame}

\begin{frame}
\frametitle{{\tt parallel} Once}
\begin{itemize}
\item We would like to avoid the overheads in creating and destroying
  threads, so we put a {\tt parallel} in front of the first for loop.
\item The {\em entire} two level loop is run by all threads.
\item Then we share the workload of the second for loop to improve
  performance,  since  most work is done in the second loop.
\end{itemize}
\end{frame}

\begin{frame}
\programlistingfirst{prime-inner.c}{}{\small}{main}{count}
\end{frame}

\begin{frame}
\frametitle{Efficient}
\begin{itemize}
\item All threads run the first loop, and share the second loop.
\item It is OK for all threads to run the first for loop
  simultaneously, since they will synchronize at every second loop.
\item Index variables {\tt i} and {\tt j} are private.
\end{itemize}
\end{frame}

\begin{frame}
\frametitle{Spawn and Join}
\centerline{\pgfimage[width=1.0\textwidth]{threads-one-spawn-join.pdf}}
\begin{itemize}
\item Multi-thread for both outer and inner loops.
\item The work in the outer loop is duplicated, but it is more
  efficient than creating and joining threads.
\end{itemize}
\end{frame}



\begin{frame}
  \frametitle{Demonstration}
  \begin{itemize}
  \item Run the {\tt prime-uni} program.
  \item Run the {\tt prime-omp} program.
  \item Run the {\tt prime-inner-omp} program.
  \end{itemize}
\end{frame}

\begin{frame}
  \frametitle{Discussion}
  \begin{itemize}
  \item Compare the execution time of all three prime counting
    programs.
  \item Is there any further optimization that can improve the
    performance?
  \end{itemize}
\end{frame}

\subsection{\tt nowait}

\begin{frame}
\frametitle{Synchronization}
\begin{itemize}
\item A for pragma will synchronize all threads before leaving the
  for statement.
\item If the for statement is still within the same {\tt parallel}
  directive, then a barrier synchronization will do.
\item If the for statement is at the end of the {\tt parallel}
  directive, threads will be joined by the master thread and
  destroyed.
\end{itemize}
\end{frame}

\begin{frame}
\frametitle{\tt nowait}
\begin{itemize}
\item Sometimes we do not wish the the threads to wait for each other.
\item Those finished earlier can go on to the next statement to improve
  performance.
\item We can only do this if the following statement does {\em not}
  depend on the previous statement.
\item In this case we can use {\tt nowait} clause.
\end{itemize}
\end{frame}

\begin{frame}
\frametitle{\tt nowait}
\prototype{nowait.h}
\end{frame}

\begin{frame}
\frametitle{Two loops}
\begin{itemize}
\item We place two for directives with a parallel directive.
\item The first for has an ascending workload, and the second loop has
a descending workload. 
\end{itemize}
\end{frame}

\programlistingtwoslides{2for.c}{}{\footnotesize}{main}{twoloop}{end}

\begin{frame}
  \frametitle{Demonstration}
  \begin{itemize}
  \item Run the {\tt 2for-omp} program with 4 threads and 8
    iterations, and observe the timing.
  \end{itemize}
\end{frame}

\begin{frame}
  \frametitle{Discussion}
  \begin{itemize}
  \item Describe the the execution time of {\tt 2for-omp} and make
    sure that it is reasonable.
  \end{itemize}
\end{frame}

\begin{frame}
\frametitle{Wait}
\begin{itemize}
\item In the previous program If the work of the second loop does not
  depend on the first loop, then we can let the threads go to the
  second directly.
\item Since the first for has an {\em ascending} workload, and the
  second loop has a {\em descending} workload, if we let the threads
  to go to the second loop then the workload imbalance will be
  reduced.
\item We only need to add a {\tt nowait clause} at the first {\tt for}
  directive.
\end{itemize}
\end{frame}

\begin{frame}
\programlistingfirst{2for-nowait.c}{}{\footnotesize}{twoloop}{end}
\end{frame}

\begin{frame}
  \frametitle{Demonstration}
  \begin{itemize}
  \item Run the {\tt 2for-nowait-omp} program with 4 threads and 8
    iterations, and observe the timing.
  \end{itemize}
\end{frame}

\begin{frame}
  \frametitle{Discussion}
  \begin{itemize}
  \item Describe the the execution time of {\tt 2for-omp} and make
    sure that it is reasonable.
  \end{itemize}
\end{frame}

\begin{frame}
\frametitle{\tt nowait}
\begin{itemize}
\item We consider our previous prime number counting program.
\item Previously the second marking {\tt for} will synchronize before
  going back to the outside {\tt for}.
\item We would like to remove this synchronization, and let each
  thread start the ``prime'' finding as soon as possible.
\item This will not cause a race condition since the threads will
  still synchronize at the beginning of the next inner loop.
\end{itemize}
\end{frame}

\begin{frame}
\programlistingfirst{prime-inner-nowait.c}{nowait}{\small}{count}{end}
\end{frame}

\begin{frame}
  \frametitle{Demonstration}
  \begin{itemize}
  \item Run {\tt prime-inner-omp} and {\tt prime-inner-nowait-omp} and
    compare their execution time.
  \end{itemize}
\end{frame}

\begin{frame}
  \frametitle{Discussion}
  \begin{itemize}
  \item Why can the {\tt nowait} clause improve performance?
  \end{itemize}
\end{frame}


\subsection{Critical Region}

\begin{frame}
  \frametitle{Compute $\pi$} 
  \begin{itemize}
  \item We calculate $\pi$ by integrating $f(x) = {4 \over {1 +
      x^2}}$, where $x$ is from 0 to 1.
  \item Divide the interval into $N$ pieces, and assume the area in each
    interval is a trapezoid, then sum the area of these $N$ trapezoids
    into a variable {\tt area}.  
  \item Note that in each interval you only need to compute $4 \over {1
    + x^2}$ once.
  \end{itemize}
\end{frame}

\begin{frame}
  \frametitle{Parallel Version} 
  \begin{itemize}
    \item We use {\tt x} to denote the x coordinate, and {\tt area}
      for the area in the integral.
    \item We parallelize the program by adding {\tt parallel for} pragma
      and declare {\tt x} as private.
  \end{itemize}
\end{frame}

\begin{frame}
  \programlisting{pi.c}{Compute $\pi$}{\scriptsize}
\end{frame}

\begin{frame}
  \frametitle{Demonstration}
  \begin{itemize}
  \item Run {\tt pi-omp}.
  \end{itemize}
\end{frame}

\begin{frame}
  \frametitle{Discussion}
  \begin{itemize}
  \item Is the answer from the parallel version correct? 
  \item Find out why the sequential version will not compile and fix
    it.  Then compare the the execution time of these two programs.
  \end{itemize}
\end{frame}

\begin{frame}
  \frametitle{Atomic} 
  \begin{itemize}
  \item We did not declare {\tt area} private because it has the final
    global answer.
  \item As a result the operation on {\tt area} must be {\em atomic}
    to avoid race condition.
  \item We simply use {\tt critical} directive on the loop body to
    ensure atomic condition.
  \end{itemize}
\end{frame}

\begin{frame}
  \programlistingfirst{pi-critical.c}{Compute $\pi$}{\footnotesize}{loop}{end}
\end{frame}

\begin{frame}
  \frametitle{Demonstration}
  \begin{itemize}
  \item Run the {\tt pi-critical-omp} program.
  \end{itemize}
\end{frame}

\begin{frame}
  \frametitle{Discussion}
  \begin{itemize}
  \item Is the answer from the parallel version correct? 
  \item Compare the the execution time of with the two previous
    programs.
  \end{itemize}
\end{frame}

\begin{frame}
  \frametitle{Atomic} 
  \begin{itemize}
  \item It appears that we do not need to make the entire loop body
    critical because {\tt x} is already private.
  \item We now only use critical directive on the area summation.
  \end{itemize}
\end{frame}

\begin{frame}
  \frametitle{Critical Section}
  \begin{itemize}
  \item It is essential to reduce the size of the critical section,
  \item We want a thread to get through a critical section as soon as
    possible, so that other threads can get into the critical section
    as well.
\end{itemize}
\end{frame}

\begin{frame}
  \programlistingfirst{pi-critical-small.c}{Compute $\pi$}{\footnotesize}{loop}{end}
\end{frame}

\begin{frame}
  \frametitle{Demonstration}
  \begin{itemize}
  \item Run the {\tt pi-critical-small-omp} program.
  \end{itemize}
\end{frame}

\begin{frame}
  \frametitle{Discussion}
  \begin{itemize}
  \item Is the answer from the parallel version correct? 
  \item Compare the the execution time of with the three previous
    programs.
  \end{itemize}
\end{frame}

\begin{frame}
  \frametitle{Private Variables} 
  \begin{itemize}
  \item The performance improvement is very limited by a smaller
    critical section because it is not very different from the
    previous implementation.
    \begin{itemize}
      \item Two statements v.s. one statement.
    \end{itemize}
    \item If there are many statements in the loop body the benefit
      will more obvious.
  \end{itemize}
\end{frame}

\begin{frame}
  \frametitle{Critical Section}
  \begin{itemize}
  \item The number of critical sections is enormous.
  \item We would like to remove these time consuming critical sections
    by letting all threads to work on its integrals by summing into
    its own {\tt area}.
  \item The idea is to use a global array to store the individual {\tt
    area}, then the master threads can add sum them up.
  \end{itemize}
\end{frame}

\begin{frame}
\frametitle{Array Implementation}
\begin{itemize}
\item We need an array for threads to store its area.
\item During each iteration a thread needs to call {\tt
  omp\_get\_thread\_num()} to know where to store its area, which is a
  significant overheads.
\end{itemize}
\end{frame}

\begin{frame}
\programlistingfirst{pi-array.c}{}{\small}{main}{end}
\end{frame}

\begin{frame}
  \frametitle{Demonstration}
  \begin{itemize}
  \item Run the {\tt pi-array-omp} program.
  \end{itemize}
\end{frame}

\begin{frame}
  \frametitle{Discussion}
  \begin{itemize}
  \item Is the answer from the parallel version correct? 
  \item Compare the the execution time of with the previous programs.
  \item Is there any way to reduce the overheads in calling {\tt
    omp\_get\_thread\_num()}?
  \end{itemize}
\end{frame}

\begin{frame}
  \frametitle{Reduction}
  \begin{itemize}
  \item We now use a reduction to compute the integral.
  \item The implementation is much cleaner and (hopefully) with better
    performance because it has been optimized by the OpenMP library.
  \end{itemize}
\end{frame}

\begin{frame}
  \programlistingfirst{pi-reduction.c}{Compute $\pi$}{\footnotesize}{loop}{end}
\end{frame}

\begin{frame}
  \frametitle{Demonstration}
  \begin{itemize}
  \item Run the {\tt pi-reduction-omp} program.
  \end{itemize}
\end{frame}

\begin{frame}
  \frametitle{Discussion}
  \begin{itemize}
  \item Is the answer from the parallel version correct? 
  \item Compare the the execution time of with the previous
    programs.
  \end{itemize}
\end{frame}

\subsection{Double Buffer}

\begin{frame}
  \frametitle{Game of Life}
  \begin{itemize}
  \item A two-dimensional board with cells.  A cell could either {\em
    live} or {\em dead}.
  \item The status of a cell evolves according to its status and the
      status of  its eight neighbors.
  \item A dead cell with exactly {\em three} live neighbors becomes
    a live cell. 
  \item A live cell with two or three live neighbors stays live,
    otherwise it becomes dead.
  \end{itemize}
\end{frame}

\begin{frame}
  \frametitle{Double Buffer}
  \begin{itemize}
    \item It is intuitive that we keep the status of cells in a two
      dimensional array.
    \item Since the status of cell depends on the status of others, if
      we modify a cell directly it will affect the computation on
      other cells.  This causes ``race'' condition during the update.
    \item Instead we use {\em two} arrays $A$ and $B$ to store the
      status of cells.  In initially the status is in $A$.
  \end{itemize}
\end{frame}

\begin{frame}
  \frametitle{Iterations}
  \begin{itemize}
    \item We repeat the following steps.
      \begin{itemize}
      \item We set the cell status of $B$ according to $A$ in the
        first, third, etc. iterations.
      \item We set the cell status of $A$ according to $B$ in the
        second, fourth, etc. iterations.
      \end{itemize}
  \end{itemize}
\end{frame}

\begin{frame}
\frametitle{Double Buffers}
\begin{itemize}
\item We use two buffers {\tt A} and {\tt B} to avoid data
  inconsistency in updating the status of cells.
\item We use a macro to compute the number of live neighboring cells
  -- a live cell has 1 and a dead cell has 0.
\item We pad the broad boundary so that we can use a single macro to
  compute the number of live neighboring cells
\end{itemize}
\end{frame}

\begin{frame}
\programlistingfirst{life.c}{}{\footnotesize}{header}{print}
\end{frame}

\begin{frame}
\frametitle{Print}
\begin{itemize}
\item We use a simple printing routine to print the status of cells.
\end{itemize}
\end{frame}

\begin{frame}
\programlistingrest{life.c}{}{\footnotesize}{print}{main}
\end{frame}

\begin{frame}
\frametitle{Input}
\begin{itemize}
\item if the flag {\tt READINPUT} is set the main program will read the
  size of the board, the number of generations,and the cell status
  from stdin.  
\item Otherwise it will generate a random input of size 4096 by 4096,
  and repeat for ten generations.
\end{itemize}
\end{frame}

\begin{frame}
\programlistingrest{life.c}{}{\footnotesize}{main}{generation}
\end{frame}

\begin{frame}
\frametitle{Dead or Alive}
\begin{itemize}
\item Depending on the generation the program will set the cell status
  of {\tt B} with {\tt A}, or in the other direction.
\item A cell will be alive only if it is dead now and has three live
  neighbors, or it is live and it has two or three live neighbors now.
\item We simply use a {\tt parallel for} directive to distribute the
  workload.
\end{itemize}
\end{frame}

\begin{frame}
\programlistingrest{life.c}{}{\footnotesize}{generation}{printcell}
\end{frame}

\begin{frame}
\frametitle{Final Status}
\begin{itemize}
\item if the flag {\tt PRINT} is set the main program will output the
  final cell status.
\end{itemize}
\end{frame}

\begin{frame}
\programlistingrest{life.c}{}{\footnotesize}{printcell}{end}
\end{frame}

\begin{frame}
  \frametitle{Demonstration}
  \begin{itemize}
  \item Run the {\tt life-uni} program.
  \item Run the {\tt life-omp} program.
  \end{itemize}
\end{frame}

\begin{frame}
  \frametitle{Discussion}
  \begin{itemize}
  \item Compute the speedup of the parallel program.
  \item What kind of scheduling policy was used?
  \item Is there any optimization that can improve the performance?
  \end{itemize}
\end{frame}


%% \begin{frame}
%%   \frametitle{Traveling Salesman} Modify your $N$-queen program into
%%   a sequential traveling salesman program and submit it.
%%   \begin{itemize}
%%   \item The distance between two points are ${dx}^2 + {dy}^2$ (no
%%     sqrt!).
%%   \item The traveling salesman does not need to go back to the city
%%     where he started, so the solution is a {\em permutation} of all
%%     cities.
%%   \item Read the input form stdin.  The first line is the number of
%%     points and each line after has the $x$ and $y$ coordinates of a
%%     city.
%%   \item The output is the minimum total distance.
%%   \end{itemize}
%% \end{frame}

%% \begin{frame}
%%   \frametitle{Cuts} 
%%   \begin{itemize}
%%   \item Add cuts into your traveling salesman program and submit it.
%%   \item What concept do you need in order to maintain the minimum
%%     solution so far?
%%   \end{itemize}
%% \end{frame}

%% \begin{frame}
%%   \frametitle{Parallel Traveling Salesman} Now parallelize your
%%   traveling salesman program.
%%   \begin{itemize}
%%   \item What concept do you need in order to maintain the minimum
%%     solution so far? 
%%   \item Do you need to protect this variable since now we may have
%%     multiple thread accessing it? 
%%   \end{itemize}
%% \end{frame}

%% \begin{frame}
%%   \frametitle{Parallel Version} 
%%   \begin{itemize}
%%   \item Now write a parallel version by only adding {\tt parallel for}
%%     pragma.
%%   \item Make sure that the answer is {\em incorrect}.
%%   \end{itemize}
%% \end{frame}

%% \begin{frame}
%%   \frametitle{Critical pragma} 
%%   \prototype{critical}{Critical pragma}
%%   \begin{itemize}
%%   \item When a thread enters the critical section, the others could
%%     not.
%%   \item The {\tt area} is {\em protected} by a critical section.
%%   \end{itemize}
%% \end{frame}

%% \begin{frame}
%%   \frametitle{Critical pragma} Now write two parallel versions using
%%   critical section 
%%   \begin{itemize}
%%   \item The critical section covers the entire loop body to avoid
%%     race condition.
%%   \item The critical section covers the addition to the area to
%%     avoid race condition.
%%   \end{itemize}
%% \end{frame}

%% \begin{frame}
%%   \frametitle{Critical pragma} 
%%   \begin{itemize}
%%   \item Compare the timing between the two parallel versions you
%%     just wrote.  Which one is more efficient?  Why?
%%   \item Submit the one that is more efficient?
%%   \end{itemize}
%% \end{frame}

%\end{CJK}
\end{document}
