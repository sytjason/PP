
% global setting

\newcommand{\comment}[1]{}

% list environment gloabal setting

% to specify the size of fonts in lst environment

\lstset{showstringspaces=false}
\lstset{language=C}
\lstset{basicstyle=\tt}
\lstset{keywordstyle=\color{blue}}
\lstset{emphstyle=\color{red}}
\lstset{numbers=left}
\lstset{frame=single}
\lstset{rangeprefix=\/*\ }
\lstset{rangesuffix=\ *\/}
\lstset{includerangemarker=false}

%% slides related macros

\newenvironment{mytitle}[2]
{\begin{frame}{#1} \vspace{0.2\textheight}{\huge #2} \end{frame}}

%% Macro for non-numbered frames listing 
%% \nonumberlisting{title}{file}{options}

\newcommand{\nonumberlisting}[3]
{\noindent{#1}
{\lstinputlisting[#3]{#2}}}

%% Program listing macros

%% \programlisting{file}{desp}{font}
%% \programlistingfirst{file}{desp}{font}{start}{end}
%% \programlistingrest{file}{descp}{font}{start}{end}


%% List a program in one page with a font size.

\newcommand{\programlisting}[3]
{\renewcommand{\lstlistingname}{#3 \bf Example}
\lstinputlisting[basicstyle={#3 \tt},caption={#3 \bf (#1) #2}]{#1}}

%% List a program for the first page within a given range.

\newcommand{\programlistingfirst}[5]
{\renewcommand{\lstlistingname}{#3 \bf Example}
\lstinputlisting[basicstyle={#3 \tt},linerange={#4-#5},caption={#3 \bf (#1) #2}]{#1}}

%% List a program for the second page (or later).

\newcommand{\programlistingrest}[5]
{\lstinputlisting[basicstyle={#3 \tt},linerange={#4-#5}]{#1}}

%% List a program in two pages.

\newcommand{\programlistingtwoslides}[6]
{\begin{frame}                  % do not add [fragile]
\programlistingfirst{#1}{#2}{#3}{#4}{#5}
\end{frame}
\begin{frame}
\programlistingrest{#1}{#2}{#3}{#5}{#6}
\end{frame}}

%% List a program in three pages.

\newcommand{\programlistingthreeslides}[7]
{\begin{frame}                  % do not add [fragile]
\programlistingfirst{#1}{#2}{#3}{#4}{#5}
\end{frame}
\begin{frame}
\programlistingrest{#1}{#2}{#3}{#5}{#6}
\end{frame}
\begin{frame}
\programlistingrest{#1}{#2}{#3}{#6}{#7}
\end{frame}
}


%% Program input/output macros

%% \programoutput{file}
%% \programoutputfirst{file}{font}{options}
%% \programoutputrest{file}{font}{startline}{endline}
%% \programoutputtwoslides{file}{font}{start}{endl}{start2}{end2}

%% Program output macor with default setting

\newcommand{\programoutput}[1]
{\programoutputfirst{#1}{}{}}

%% Program output macor for the first page

\newcommand{\programoutputfirst}[3]
{\lstset{basicstyle={#2 \tt}}
\nonumberlisting{#2 \bf 輸出}{#1.out}{#3}}

%% Program output macor for later page

\newcommand{\programoutputrest}[4]
{\lstinputlisting[firstnumber={#3},basicstyle={\tt #2},linerange={#3-#4}]
{#1.out}}

%% Program output macor for two slides

\newcommand{\programoutputtwoslides}[6]
{\begin{frame}                  % do not add [fragile]
\programoutputfirst{#1}{#2}{linerange={#3-#4}}
\end{frame}
\begin{frame}
\programoutputrest{#1}{#2}{#5}{#6}
\end{frame}}


%% \programinput{file}
%% \programinputfirst{file}{font}{option}

%% Program input macor with default setting

\newcommand{\programinput}[1]
{\programinputfirst{#1}{}{}}

%% Program input macor for the first page

\newcommand{\programinputfirst}[3]
{\lstset{basicstyle={#2 \tt}}
\nonumberlisting{#2 \bf 輸入}{#1.in}{#3}}


%% \piotwocol{file}

\newcommand{\piotwocol}[1]{\piotwocoldetail{#1}{}{}}

\newcommand{\piotwocoldetail}[2]{
\begin{columns}[t,onlytextwidth]
\begin{column}{0.45\textwidth}
\programinputfirst{#1}{#2}{}
\end{column}
\begin{column}{0.45\textwidth}
\programoutputfirst{#1}{#2}{}
\end{column}
\end{columns}}

%% \piotworow{file}

\newcommand{\piotworow}[1]
{\programinput{#1}
\programoutput{#1}}

\newcommand{\piotworowdetail}[2]
{\programinputfirst{#1}{#2}{}
\programoutputfirst{#1}{#2}{}}


%% \newcommand{\headerfile}[1]
%% {\nonumberlisting {{\bf 標頭檔} {\tt #1}}{#1}{}}

\newcommand{\header}[2]{\headerdetail{#1}{#2}{}}

\newcommand{\headerdetail}[3]
{{\renewcommand{\lstlistingname}{#3 \bf Header}
\lstset{basicstyle={#3 \tt}}
\lstset{label=header:#1,caption={(#1) #2}}
\lstinputlisting{#1}}}

%% Macro for prototype
%% \prototype{file}

\newcommand{\prototype}[1]{\prototypedetail{#1}{}}

%% Macro for prototype with detialed setting
%% \prototypedetail{file}{font}

\newcommand{\prototypedetail}[2]
{{\renewcommand{\lstlistingname}{#2 \bf Prototype}
\lstset{basicstyle={#2 \tt}}
\lstset{frame=single,label=prototype:#1,caption={#2}}
\lstinputlisting{#1}}}

%% Macro for phrase
%% \nphrase{file}{desc}

%% \newcommand{\nphrase}[2]
%% {{\renewcommand{\lstlistingname}{\lstfontsize \bf 片語}
%% \lstset{label=phr:#1,frame=single,caption={\lstfontsize #2}}
%% \lstset{basicstyle={\lstfontsize \tt}}
%% \lstinputlisting{#1.p}}
%% }

\newcommand{\nphrase}[2]
{\nphrasedetail{#1}{#2}{}}

%% Macro for detail phrase
%% \nphrasedetail{file}{desc}{font}

\newcommand{\nphrasedetail}[3]
{{\renewcommand{\lstlistingname}{#3 \bf 片語}
\lstset{label=phr:#1,frame=single,caption={#3 #2}}
\lstset{basicstyle={#3 \tt}}
\lstinputlisting{#1.p}}}



\newcommand{\inputfile}[1]
{\nonumberlisting {{\bf 輸入檔} {\tt #1}}{#1}{}}

\newcommand{\outputfile}[1]
{\nonumberlisting {{\bf 輸出檔} {\tt #1}}{#1}{}}

% Macro for listings that are labeled.
% [1] is the filename, [2] is the description

\newcommand{\commandline}[1]
{\lstset{basicstyle={\tt}}
\nonumberlisting{\bf 命令列}{#1.bat}{}}

% {\boxedlisting{\bf 命令列}{#1.bat}}

% Macro for listings that are labeled.
% [1] is the filename, [2] is the description

\newcommand{\out}[2]
{{\renewcommand{\lstlistingname}{\bf 執行結果}
\lstset{label=out:#1,caption=#2}
\lstinputlisting{#1.out}}}




